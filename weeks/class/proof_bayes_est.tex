% Options for packages loaded elsewhere
\PassOptionsToPackage{unicode}{hyperref}
\PassOptionsToPackage{hyphens}{url}
%
\documentclass[
  ignorenonframetext,
]{beamer}
\usepackage{pgfpages}
\setbeamertemplate{caption}[numbered]
\setbeamertemplate{caption label separator}{: }
\setbeamercolor{caption name}{fg=normal text.fg}
\beamertemplatenavigationsymbolsempty
% Prevent slide breaks in the middle of a paragraph
\widowpenalties 1 10000
\raggedbottom
\setbeamertemplate{part page}{
  \centering
  \begin{beamercolorbox}[sep=16pt,center]{part title}
    \usebeamerfont{part title}\insertpart\par
  \end{beamercolorbox}
}
\setbeamertemplate{section page}{
  \centering
  \begin{beamercolorbox}[sep=12pt,center]{part title}
    \usebeamerfont{section title}\insertsection\par
  \end{beamercolorbox}
}
\setbeamertemplate{subsection page}{
  \centering
  \begin{beamercolorbox}[sep=8pt,center]{part title}
    \usebeamerfont{subsection title}\insertsubsection\par
  \end{beamercolorbox}
}
\AtBeginPart{
  \frame{\partpage}
}
\AtBeginSection{
  \ifbibliography
  \else
    \frame{\sectionpage}
  \fi
}
\AtBeginSubsection{
  \frame{\subsectionpage}
}

\usepackage{amsmath,amssymb}
\usepackage{lmodern}
\usepackage{iftex}
\ifPDFTeX
  \usepackage[T1]{fontenc}
  \usepackage[utf8]{inputenc}
  \usepackage{textcomp} % provide euro and other symbols
\else % if luatex or xetex
  \usepackage{unicode-math}
  \defaultfontfeatures{Scale=MatchLowercase}
  \defaultfontfeatures[\rmfamily]{Ligatures=TeX,Scale=1}
\fi
% Use upquote if available, for straight quotes in verbatim environments
\IfFileExists{upquote.sty}{\usepackage{upquote}}{}
\IfFileExists{microtype.sty}{% use microtype if available
  \usepackage[]{microtype}
  \UseMicrotypeSet[protrusion]{basicmath} % disable protrusion for tt fonts
}{}
\makeatletter
\@ifundefined{KOMAClassName}{% if non-KOMA class
  \IfFileExists{parskip.sty}{%
    \usepackage{parskip}
  }{% else
    \setlength{\parindent}{0pt}
    \setlength{\parskip}{6pt plus 2pt minus 1pt}}
}{% if KOMA class
  \KOMAoptions{parskip=half}}
\makeatother
\usepackage{xcolor}
\newif\ifbibliography
\setlength{\emergencystretch}{3em} % prevent overfull lines
\setcounter{secnumdepth}{-\maxdimen} % remove section numbering


\providecommand{\tightlist}{%
  \setlength{\itemsep}{0pt}\setlength{\parskip}{0pt}}\usepackage{longtable,booktabs,array}
\usepackage{calc} % for calculating minipage widths
\usepackage{caption}
% Make caption package work with longtable
\makeatletter
\def\fnum@table{\tablename~\thetable}
\makeatother
\usepackage{graphicx}
\makeatletter
\def\maxwidth{\ifdim\Gin@nat@width>\linewidth\linewidth\else\Gin@nat@width\fi}
\def\maxheight{\ifdim\Gin@nat@height>\textheight\textheight\else\Gin@nat@height\fi}
\makeatother
% Scale images if necessary, so that they will not overflow the page
% margins by default, and it is still possible to overwrite the defaults
% using explicit options in \includegraphics[width, height, ...]{}
\setkeys{Gin}{width=\maxwidth,height=\maxheight,keepaspectratio}
% Set default figure placement to htbp
\makeatletter
\def\fps@figure{htbp}
\makeatother

\makeatletter
\makeatother
\makeatletter
\makeatother
\makeatletter
\@ifpackageloaded{caption}{}{\usepackage{caption}}
\AtBeginDocument{%
\ifdefined\contentsname
  \renewcommand*\contentsname{Table of contents}
\else
  \newcommand\contentsname{Table of contents}
\fi
\ifdefined\listfigurename
  \renewcommand*\listfigurename{List of Figures}
\else
  \newcommand\listfigurename{List of Figures}
\fi
\ifdefined\listtablename
  \renewcommand*\listtablename{List of Tables}
\else
  \newcommand\listtablename{List of Tables}
\fi
\ifdefined\figurename
  \renewcommand*\figurename{Figure}
\else
  \newcommand\figurename{Figure}
\fi
\ifdefined\tablename
  \renewcommand*\tablename{Table}
\else
  \newcommand\tablename{Table}
\fi
}
\@ifpackageloaded{float}{}{\usepackage{float}}
\floatstyle{ruled}
\@ifundefined{c@chapter}{\newfloat{codelisting}{h}{lop}}{\newfloat{codelisting}{h}{lop}[chapter]}
\floatname{codelisting}{Listing}
\newcommand*\listoflistings{\listof{codelisting}{List of Listings}}
\makeatother
\makeatletter
\@ifpackageloaded{caption}{}{\usepackage{caption}}
\@ifpackageloaded{subcaption}{}{\usepackage{subcaption}}
\makeatother
\makeatletter
\@ifpackageloaded{tcolorbox}{}{\usepackage[many]{tcolorbox}}
\makeatother
\makeatletter
\@ifundefined{shadecolor}{\definecolor{shadecolor}{rgb}{.97, .97, .97}}
\makeatother
\makeatletter
\makeatother
\ifLuaTeX
  \usepackage{selnolig}  % disable illegal ligatures
\fi
\IfFileExists{bookmark.sty}{\usepackage{bookmark}}{\usepackage{hyperref}}
\IfFileExists{xurl.sty}{\usepackage{xurl}}{} % add URL line breaks if available
\urlstyle{same} % disable monospaced font for URLs
\hypersetup{
  pdftitle={Bayes estimator under absolute loss},
  hidelinks,
  pdfcreator={LaTeX via pandoc}}

\title{Bayes estimator under absolute loss}
\author{}
\date{}

\begin{document}
\frame{\titlepage}
\ifdefined\Shaded\renewenvironment{Shaded}{\begin{tcolorbox}[breakable, frame hidden, borderline west={3pt}{0pt}{shadecolor}, sharp corners, enhanced, interior hidden, boxrule=0pt]}{\end{tcolorbox}}\fi

\begin{frame}{Theorem}
\protect\hypertarget{theorem}{}
Under absolute loss \(L(\theta, a) = |\theta - a|\), a Bayes estimator
for \(\theta\) is any posterior median of \(\theta\).

\begin{itemize}[<+->]
\item
  That is, a Bayes estimator is a value \(\delta(\mathbf{X}) \equiv m\)
  such that \(\text{Pr}(\theta \leq m | \mathbf{x}) \leq \frac{1}{2}\)
  and \(\text{Pr}(\theta \geq m | \mathbf{x}) \leq \frac{1}{2}\).
\item
  Note that when \(\theta\) is continuous, there exists a single median.
\end{itemize}
\end{frame}

\begin{frame}{Proof set-up}
\protect\hypertarget{proof-set-up}{}
\begin{itemize}[<+->]
\item
  Note that the absolute value function is not differentiable, so
  proving a maximum/minimum cannot rely on derivatives!
\item
  Assume that \(\theta\) is continuous, so that if \(m\) is the
  posterior median, then
  \(\text{Pr}(\theta \geq m | \mathbf{x}) = \frac{1}{2} = \text{Pr}(\theta \leq m | \mathbf{x})\).
\item
  Let \(a\) be any other estimator of \(\theta\). Assume \(a < m\). (The
  proof for \(a > m\) is very similar.)
\item
  We will show that

  \[
  \mathbb{E}[L(\theta , a) | \mathbf{x}] - \mathbb{E}[L(\theta, m) | \mathbf{x}]  \geq 0
  \]

  thus demonstrating that \(m\) minimizes the expected loss.
\end{itemize}
\end{frame}

\begin{frame}{Proof}
\protect\hypertarget{proof}{}
\[
\begin{align}
\mathbb{E}[L(\theta , a) | \mathbf{x}] & - \mathbb{E}[L(\theta, m) | \mathbf{x}] = \int |\theta - a| p(\theta | \mathbf{x}) d\theta - \int |\theta - m| p(\theta | \mathbf{x})d\theta \\ 
&= \int \left(|\theta - a| - |\theta - m|\right) p(\theta | \mathbf{x})d\theta  \\
&= \int_{-\infty}^{a} \left(|\theta - a| - |\theta - m|\right) p(\theta | \mathbf{x})d\theta + \int_{a}^{m} \left(|\theta - a| - |\theta - m|\right) p(\theta | \mathbf{x})d\theta + \int_{m}^{\infty} \left(|\theta - a| - |\theta - m|\right) p(\theta | \mathbf{x})d\theta \\
&= \int_{-\infty}^{a} ((a-\theta) - ( m - \theta)) p(\theta | \mathbf{x}) d\theta + \int_{a}^{m} ((\theta - a) - (m- \theta)) p(\theta | \mathbf{x}) d\theta  +
\int_{m}^{\infty} ((\theta - a) - (\theta - m)) p(\theta | \mathbf{x}) d\theta \\
&= \int_{-\infty}^{a} (a - m)  p(\theta | \mathbf{x}) d\theta + \int_{a}^{m} (2\theta - a- m)  p(\theta | \mathbf{x}) d\theta + \int_{m}^{\infty} (m-a) p(\theta | \mathbf{x}) d\theta \\
& \color{orange}{\geq}  \int_{-\infty}^{a} (a - m)  p(\theta | \mathbf{x}) d\theta + \int_{a}^{m} (2\color{orange}{a} - a- m)  p(\theta | \mathbf{x}) d\theta + \int_{m}^{\infty} (m-a) p(\theta | \mathbf{x}) d\theta \\
&= (a-m)\text{Pr}(\theta \leq a | \mathbf{x}) + \color{purple}{(a-m)\text{Pr}(a  < \theta \leq m | \mathbf{x}) }+ (m-a) \text{Pr}(\theta \geq m | \mathbf{x}) \\
&= (a-m)\text{Pr}(\theta \leq a | \mathbf{x}) + \color{purple}{(a-m) \text{Pr}(\theta \leq m | \mathbf{x})  - (a-m) \text{Pr}(\theta \leq a | \mathbf{x})} + (m-a) \text{Pr}(\theta \geq m | \mathbf{x})  \\
&= (a-m) \text{Pr}(\theta \leq m | \mathbf{x}) +  (m-a) \text{Pr}(\theta \geq m | \mathbf{x}) \\
&= (a-m)\left(\frac{1}{2}\right) + (m-a)\left(\frac{1}{2}\right) \\
&= (a-m)\left(\frac{1}{2}\right) - (a-m)\left(\frac{1}{2}\right)\\
&= 0
\end{align}
\]
\end{frame}



\end{document}
